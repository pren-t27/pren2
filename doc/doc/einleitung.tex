\section{Einleitung}
Die Module PREN1 und PREN2 (Produktentwicklung) sind Teil des Studiums einer 
technischen Fachrichtung an der Hochschule Luzern Technik \& Architektur 
(HSLU T\&A). 
Zu Beginn des Moduls werden die Studierenden in interdisziplinäre Teams 
eingeteilt. Diese Teams umfassen Studierende der Studiengänge Maschinenbau, 
Informatik und Elektrotechnik. Die Teams erhalten am ersten Tag eine Aufgabe. 
Diese muss in den zwei darauffolgenden Semestern gelöst werden. 
Die Aufgabe für das Herbst- und Frühlingssemester 2014/2015 ist es, ein System 
zu entwickeln, das in der Lage ist, fünf Bälle in einen Korb zu befördern. 
Einschränkungen bilden dabei die Grösse des Spielfelds und die Bedingung, dass 
die Bälle entweder geworfen oder geflogen werden müssen. 
Für weitere Details, siehe Abschnitt \ref{sec:aufgabe} \nameref{sec:aufgabe}. 

\noindent Um diese Aufgabe zu meistern wurde das System in einzelne Teile 
separiert. Jedes dieser Teilsysteme hat die Erfüllung eines Teilauftrages als 
Ziel. Anhand der Ideensammlung wurden möglichst viele Lösungsansätze gesammelt 
und aufgelistet. Danach folgt eine Technologierecherche, deren Ziel es ist, 
die geeignetsten Technologien und Standards zu finden. Aus den gesamten 
Recherchen werden Lösungskonzepte erstellt, die eine Kombination der vorher 
gesammelten Ideen darstellen. All diese Lösungskonzepte sollen theoretisch die 
Aufgabenstellung sowie die Produktanforderungen erfüllen können. Das aus den 
Lösungskonzepten ausgewählte Hauptkonzept wird im Frühlingssemester 2015 im 
Modul PREN2 realisiert.
