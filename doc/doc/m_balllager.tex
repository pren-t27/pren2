\subsection{Balllager}
Das Balllager dient als Grundstruktur, auf welcher der Ballnachschub und der Motor befestigt sind. Gelagert werden die Bälle im Inneren des quadratischen Querschnittes. Die Halterungen des Motors wird jeweils auf 
beiden Aussenseiten des vorderen Endes befestigt. Damit das Stahlband des Ballnachschubs optimal aufgewickelt werden kann, werden die herausstehenden Enden der Nieten mit Aluminiumleisten abgedeckt.

Anhand der Erfahrungen erster Testläufe wird entschieden, keine Verstärkung an der Unterseite des Vorderen Endes anzubringen. Diese Verstärkung hätte eine zu starke Durchbiegung des Blechs beim Abschuss der Bälle verhindert.
