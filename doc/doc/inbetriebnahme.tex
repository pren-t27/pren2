\section{Inbetriebnahme}
Hier werden die Inbetriebnahmetests beschrieben.

\subsection{Tests Einzelkomponenten}

Kurz nach der Endmontage werden die einzelnen Komponenten getestet. Eine erste Inbetriebnahme des Ballnachschubs zeigt, das die Funktion gewährleistet ist. Das Drehmoment des Servomotors genügt um die 5 Bälle nach oben zu befördern. Die Geschwindigkeit des Ballnachschubs kann mit der Betriebsspannung des Motors eingestellt werden, allerdings ist sie auf ca. 12V begrenzt, da anderenfalls der Motor überlastet wird.

Der BLDC Motor kann bereits angesteuert werden und über ein Netzgerät wird die Drehzahl eingestellt. Hier zeigt sich, dass der Pneu eine Unwucht hat. Leider kann diese Unwucht nicht mit konventionellen Massnahmen behoben werden, das es sich um Ungenauigkeiten in der Dicke und in der Form handelt. Mit der erforderten Drehzahl für die Schussweite von ca. 2m wirkt sich die Unwucht jedoch nicht auf die Konstruktion aus.

Mit dieser Konfiguration werden die ersten Ballwürfe gemacht. Zuerst wird der Ballnachschub sehr langsam eingestellt, so dass die Drehzahl des Motors nach einem Schuss wieder auch die Anfangsdrehzahl gestiegen ist. Da die Streuung der Reichweite ungewöhnlich stark variiert (ca. 0.5m), haben wir auf der Gegenseite des Rades ein Stück Gummi aufgeklebt, so dass die Reibung des Tennisballs grösser ist. So wird die Streuung auf ca. 0.2m verkleinert. 

\begin{figure}[h!]          
	\centering             
	\includegraphics[width=0.5\textwidth]{fig/Bild_Gummi.JPG}
	\caption{Gummi zur Erhöhung der Reibung}
	\label{fig:Gummi}        
\end{figure}

In weiteren Tests werden die Bälle möglichst schnell nacheinander geschossen, so dass sich die Drehzahl des BLDC Motors nicht mehr ganz erholen kann. Komischerweise wird so die Streuung noch kleiner (nicht mehr von Auge abschätzbar). Drei weitere Tests mit Wettbewerbsbedingungen (Distanz und Höhe des Korbes) und mit schnellem Ballnachschub zeigen, dass die Genauigkeit sehr hoch ist, alle Bälle landen im Korb. Gegebenenfalls könnte mit einem anderen Pneu die Genauigkeit noch erhöht werden.

Die Drehvorrichtung funktioniert, und ist sehr schnell. Bei ersten Tests (ohne Tennisbälle) konnte innerhalb einer Sekunde mehr als eine Umdrehung gemacht werden. Schlussendlich muss sich der Turm nur um ca. 16$^\circ$ drehen, somit sollten ca. 1/5 Sekunde reichen um die Richtung einzustellen.



\subsection{Tests Gesamtsystem}
\subsubsection{Ablauf}
Die Tests des Gesamtsystems wurden durchgeführt, um das Zusammenspiel und richtige Funktionieren der einzelnen Komponenten zu gewährleisten. Dazu wird eine Bluetooth-Verbindung mit dem Gerät hergestellt. Da der BLDC-Motor beim Wettkampf bereits gestartet sein darf, wird dieser mittels Eingabefeld der Applikation bereits vor dem Wettkampfstart auf eine Drehzahl von \[\frac{1400\,Umdrehungen}{Minute}\] konfiguriert.

Anschliessend wird mittels Bilderkennung ein aktuelles Foto des Spielfelds geschossen. Auf diesem Foto wird per Mausziehen der Bereich des Spielfeldes ausgewählt. Wird nun das Startsignal gegeben, kann man einen Button drücken, der dann die Ausrichtung des Geräts übernimmt und den Ballnachschub nach einer Verzögerung von \[\frac{Anzahl\,Ticks}{20}\] in Bewegung setzt.

Sind alle Bälle geworfen worden, wird ein String an alle Observer des BluetoothReceivers gesendet. Enthält dieser String "fin", so heisst dies, dass das Gerät die Bälle alle verschossen hat und somit das Endsignal ausgelöst werden kann. Nach dem Endsignal ist der Test beendet.

\subsubsection{Resultate}

Die ersten Tests haben gezeigt, dass die Ballnachführung nicht zeitgleich mit dem Ausrichten in Bewegung gesetzt werden darf, da ansonsten der erste Ball nicht trifft. Integrationstests dieser beiden Komponenten haben gezeigt, dass die optimale Verzögerungszeit für den Start der Ballnachführung nach dem Ausrichten bei \[\frac{Anzahl\,Ticks}{20}\] liegt. 

Weitere Tests haben gezeigt, dass bei der Anfangs konfigurierten Drehzahl von 1372 pro Minute der erste Ball bei einem \[Ausrichtungswinkel > 16^\circ\] nicht trifft. Dies liegt daran, dass der erste Ball mit einer niedrigeren Geschwindigkeit an das Rad geführt wird, als die anderen Bälle. Dieses Problem ist nun behoben worden, indem der BLDC-Motor anfangs auf \[\frac{1400\,Umdrehungen}{Minute}\] eingestellt wird und zeitgleich zum Ausrichten auf \[\frac{1372\,Umdrehungen}{Minute}\] heruntergefahren wird. Tests mit dieser Konfiguration haben seither alle Bälle versenkt.