\subsection{Ballnachschub}
Der Ballnachschub besteht aus einer Halterung, Trommel, Stahlband und einem 
Servomotor. Das Stahlband wird im Inneren des Balllagers um die Bälle herum 
ausgelegt. Durch das Aufwickeln des Bandes auf die Trommel werden die Bälle 
Richtung Beschleunigungsrad gezogen. Der Antrieb erfolgt durch einen 
umgebauten Servomotor, welcher mittels Zahnriemen mit der Welle der Trommel 
verbunden ist.
Die Welle der Trommel hat einen zu grossen Durchmesser und muss daher von Hand auf das gewünschte Mass geschliffen werden.

\subsubsection{Auslegung}
Als Servo dient ein HS85MG der Firma Hitec. Dieses ist wie folgt spezifiziert: 
\begin{table}[h!]
    \centering
    \begin{zebratabular}{lll}
        \rowcolor{gray}
        Parameter &
            Wert bei 4.8\si{\volt} &
            Wert bei 6.0\si{\volt} \\
        Stellgeschwindigkeit (60\si{\degree}) &
            0.16\si{\second} &
            0.14\si{\second} \\
        Drehmoment &
            3.0\si{\kilogram\per\centi\metre} &
            3.5\si{\kilogram\per\centi\metre} \\
    \end{zebratabular}
    \caption{Spezifikation Servomotor}
\end{table}
Dies ergibt folgende Spezifikationen für den sich daraus ergebenden DC Motor: 
\begin{table}[h!]
    \centering
    \begin{zebratabular}{lll}
        \rowcolor{gray}
        Parameter &
            Wert bei 4.8\si{\volt} &
            Wert bei 6.0\si{\volt} \\
        Stellgeschwindigkeit (60\si{\degree}) &
            62.5\si{\per\minute} &
            71.4\si{\per\minute} \\
        Drehmoment &
            0.3\si{\newton\metre} &
            0.35\si{\newton\metre} \\
    \end{zebratabular}
    \caption{Spezifikation DC Motor}
\end{table}

