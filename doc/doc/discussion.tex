\section{Schlussdiskussion}
Wie bereits im Modul PREN1 stand für einen erfolgreichen Abschluss die 
Teamarbeit und ein gutes Zusammenspiel der unterschiedlichen Disziplinen im 
Vordergrund. Dem Team ist es gelungen trotz einiger Schwierigkeiten einen 
guten Weg zu finden um das geforderte Produkt zu fertigen. Hierbei konnte 
sowohl von den anderen Fachrichtungen sowie auch innerhalb der eigenen 
Disziplin neue Erfahrungen gesammelt werden. Die in der Gruppe herrschende 
Stimmung wurde als kollegial und lösungsorientiert empfunden. Diese 
Voraussetzungen ermöglichten optimale Arbeitsbedingungen und ein effektives 
Vorankommen.\\

\noindent
Das Produkt entspricht grösstenteils  dem in PREN1 ausgearbeitetem Entwurf. 
Durch frühzeitige Funktionstests und den damit verbundenen neuen Erkenntnissen 
während dem Zusammenbau, konnte das Produkt kontinuierlich verbessert werden 
und bietet so gegenüber dem ursprünglichen Entwurf einige Vorteile. Leider 
erforderten einige Funktionstests das Vorhandensein von Komponenten anderer 
Disziplinen, was teilweise zu Verzögerungen führte\\

\noindent
Einmal mehr zeigte sich, dass eine gute Planung des Projektablaufes und eine 
Strukturierung der noch bearbeiteten Aufgaben essenziell für das Gelingen 
eines Projektes sind. Durch ein regelmässiges Treffen der Gruppe und bei 
Bedarf einberufenen Sitzungen, bei welchen sämtliche Teammitglieder anwesend 
waren, konnte auch diese Schwierigkeit vorteilhaft bewältigt werden. Aufgrund 
von diesem reibungslosen Informationsaustausch konnten eigene Pendenzen und 
deren Relevanz für andere Teammitglieder eingeschätzt und die Planung 
verbessert werden.\\

\noindent
Abschliessend erhofft sich das Team 27 für den kommenden Wettbewerb ein gutes 
Gelingen und eine störungsfreie Ausführung der Aufgabe.\\
