\subsection{Energieversorgung}
Der Schrittmotor und der BLDC Motor werden mit einer Spannung von 48\si{\volt} 
betrieben. Der DC Motor für die Ballnachführung und die Kontrollplatine werden 
mit einer Spannung von 5\si{\volt} versorgt. Dazu werden Netzteile aus Servern 
verwendet. Davon werden vier in Serie geschaltet um die notwendige Spannung 
von 48\si{\volt} zu erreichen. Um Kurzschlüsse zu vermeiden, müssen die 
Netzteile potentialfrei gemacht werden. Bei den verwendeten Netzteilen vom Typ 
??? von HP wird die Sekundärseite mit einer Schraube elektrisch mit dem 
Gehäuse verbunden. Das Gehäuse ist mit PE verbunden und somit geerdet. Diese 
Schraube wird durch eine Kunststoffschraube ersetzt. Ausserdem ist der Abstand 
vom Gehäuse zur Platine an der Austrittsstelle der Platine sehr klein und kann 
zu unerwünschten Kurzschlüssen führen. Daher wird mit einem isolierenden 
Streifen aus FR4 eine Berührung verhindert. 
