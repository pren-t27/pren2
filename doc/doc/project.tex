\section{Projektmanagement}
Das Modul PREN ist ein interdisziplinäres Modul. Von jeder der Fachrichtungen 
Informatik, Elektrotechnik und Maschinenbau ist mindestens eine Person pro 
Gruppe vertreten. Das Team 27 setzt sich wie folgt zusammen:
\begin{itemize}
    \item 3 Informatiker
    \item 3 Maschinenbauer
    \item 1 Elektrotechniker
\end{itemize}
Zu Beginn des Moduls wurden den Mitgliedern des Teams verschiedene Funktionen 
zugewiesen, um eine Struktur und somit Ansprechpersonen innerhalb des Teams zu 
erhalten. Ersichtlich ist dies in folgendem Organigramm:
\newcommand\orgnode[6][]{
    \node[#1] at (#2) {
        \boxed{
            \parbox{40pt} {%
                \includegraphics[width=40pt]{#3}}%
            \parbox{10pt}{\mbox{}}%  
            \parbox{\dimexpr4.0cm-30pt\relax}{
                \textbf{#4}\\
                [.4ex]\textit{#5}\\
                [.4ex]#6\\}%
        }
    };
}
\begin{figure}[h!]
    % from http://tex.stackexchange.com/questions/170154/organisation-chart-in-latex-using-tikz
    \centering
    \begin{tikzpicture}
        \orgnode
            {0,0}
            {../fig/takuonen.png}
            {Peter Kuonen}
            {Informatik}
            {Projektleiter}
        \orgnode
            {-6,-4}
            {../fig/tawinz.png}
            {Daniel Winz}
            {Elektrotechnik}
            {Bereichsleiter E}
        \orgnode
            {0,-4}
            {../fig/mamaisse.png}
            {Andriu Maissen}
            {Maschinenbau}
            {Bereichsleiter M}
        \orgnode
            {6,-4}
            {../fig/taneidha.png}
            {Simon Neidhart}
            {Informatik}
            {Bereichsleiter I}
        \orgnode
            {0,-8}
            {../fig/tfkueng.png}
            {Yannik Küng}
            {Maschinenbau}
            {Projektmitarbeiter}
        \orgnode
            {0,-12}
            {../fig/tfmathis.png}
            {Daniel Mathis}
            {Maschinenbau}
            {Projektmitarbeiter}
        \orgnode
            {6,-8}
            {../fig/tcwespi.png}
            {Kevin Wespi}
            {Informatik}
            {Protokollführer}
        \draw[-latex, thick] (0,-1) -- (0,-2) -- (-6,-2) -- (-6,-3);
        \draw[-latex, thick] (0,-1) -- (0,-2) -- (0,-2) -- (0,-3);
        \draw[-latex, thick] (0,-1) -- (0,-2) -- (6,-2) -- (6,-3);
        \draw[-latex, thick] (-2.4,-4) -- (-2.9,-4) -- (-2.9,-8) -- (-2.4, -8);
        \draw[-latex, thick] (-2.4,-4) -- (-2.9,-4) -- (-2.9,-12) -- (-2.4, -12);
        \draw[-latex, thick] (3.6,-4) -- (3.1,-4) -- (3.1,-8) -- (3.6, -8);
    \end{tikzpicture}
    \caption{Organigramm}
    \label{fig:Organigramm}
\end{figure}

\subsection{Projektplanung}

\subsection{Finanzen und Maschinenstunden}

Für das ganze Projekt steht ein Budget von CHF 600,- zur Verfügung. Da alle Blechteile von Daniel Mathis gefertigt wurden, und für das Biegen nur eine symbolische Entschädigung von CHF 40,- anfiel, musste das Budget nicht vollständig ausgeschöpft werden.
\begin{table}[h!]
	\centering
	\begin{zebratabular}{llll}
		\rowcolor{gray}
		Lieferant & Bezeichnung & Preis in CHF & Bezahlt \\
		Conrad Electronic &	Axial Rillenkugellager 51104 & 8.95 & Yannik \\
		Microspot & LOGITECH HD Webcam C525	& 41.85 & Kevin\\
		Diverse	& Elektrotechnik & 180.00 &	Daniel W.\\
		Profiblech Kägiswil	& Blechteile biegen (Freundschaftspreis) & 40.00 & Daniel M.\\
		& Rohmaterial Alublech 1m$^2$ & 25.00 & gesponsert \\
		Arthur Weber AG & Nieten	& 11.90	& Daniel W. \\
		HSLU & Schrittmotor QMot.eu 4218-35-10-027 & 30.00 & - \\
		& TOTAL & 400.80 &
		
	\end{zebratabular}
	\caption{Finanzen Pren2}
\end{table}

Von den 10h Fertigungszeit die uns für das Projekt zur Verfügung stehen, wurden 9.05 Stunden gebraucht. Folgende Teile wurden gefertigt:

\begin{table}[h!]
	\centering
	\begin{zebratabular}{llll}
		\rowcolor{gray}
		Teilename & Stückzahl & Material & Aufwand in h \\
		Bodenplatte & 1 & Aluminium & 3\\
		Bein Adapter & 3 & Aluminium & 2\\
		Verbindung Bein-Platte & 3 & Aluminium & 1.8\\
		Achshalter & 1 & Aluminium & 0.75\\
		Hülse Servo & 2 & Aluminium & 0.25\\
		Motorhalter & 2 & Aluminium & 1.25\\
		& & TOTAL & 9.05\\
		
	\end{zebratabular}
	\caption{Fertigungszeit}
\end{table}

Die drei Füsse wurden 3D gedruckt, dafür wurden nur 0.8 Stunden benötigt von den insgesamt 25 Stunden die uns zur Verfügung stehen.
