\section{Projektmanagement}
Das Modul PREN ist ein interdisziplinäres Modul. Von jeder der Fachrichtungen 
Informatik, Elektrotechnik und Maschinenbau ist mindestens eine Person pro 
Gruppe vertreten. Das Team 27 setzt sich wie folgt zusammen:
\begin{itemize}
    \item 3 Informatiker
    \item 3 Maschinenbauer
    \item 1 Elektrotechniker
\end{itemize}
Zu Beginn des Moduls wurden den Mitgliedern des Teams verschiedene Funktionen 
zugewiesen, um eine Struktur und somit Ansprechpersonen innerhalb des Teams zu 
erhalten. Ersichtlich ist dies in folgendem Organigramm:
\newcommand\orgnode[6][]{
    \node[#1] at (#2) {
        \boxed{
            \parbox{40pt} {%
                \includegraphics[width=40pt]{#3}}%
            \parbox{10pt}{\mbox{}}%  
            \parbox{\dimexpr4.0cm-30pt\relax}{
                \textbf{#4}\\
                [.4ex]\textit{#5}\\
                [.4ex]#6\\}%
        }
    };
}
\begin{figure}[h!]
    % from http://tex.stackexchange.com/questions/170154/organisation-chart-in-latex-using-tikz
    \centering
    \begin{tikzpicture}
        \orgnode
            {0,0}
            {../fig/takuonen.png}
            {Peter Kuonen}
            {Informatik}
            {Projektleiter}
        \orgnode
            {-6,-4}
            {../fig/tawinz.png}
            {Daniel Winz}
            {Elektrotechnik}
            {Bereichsleiter E}
        \orgnode
            {0,-4}
            {../fig/mamaisse.png}
            {Andriu Maissen}
            {Maschinenbau}
            {Bereichsleiter M}
        \orgnode
            {6,-4}
            {../fig/taneidha.png}
            {Simon Neidhart}
            {Informatik}
            {Bereichsleiter I}
        \orgnode
            {0,-8}
            {../fig/tfkueng.png}
            {Yannik Küng}
            {Maschinenbau}
            {Projektmitarbeiter}
        \orgnode
            {0,-12}
            {../fig/tfmathis.png}
            {Daniel Mathis}
            {Maschinenbau}
            {Projektmitarbeiter}
        \orgnode
            {6,-8}
            {../fig/tcwespi.png}
            {Kevin Wespi}
            {Informatik}
            {Protokollführer}
        \draw[-latex, thick] (0,-1) -- (0,-2) -- (-6,-2) -- (-6,-3);
        \draw[-latex, thick] (0,-1) -- (0,-2) -- (0,-2) -- (0,-3);
        \draw[-latex, thick] (0,-1) -- (0,-2) -- (6,-2) -- (6,-3);
        \draw[-latex, thick] (-2.4,-4) -- (-2.9,-4) -- (-2.9,-8) -- (-2.4, -8);
        \draw[-latex, thick] (-2.4,-4) -- (-2.9,-4) -- (-2.9,-12) -- (-2.4, -12);
        \draw[-latex, thick] (3.6,-4) -- (3.1,-4) -- (3.1,-8) -- (3.6, -8);
    \end{tikzpicture}
    \caption{Organigramm}
    \label{fig:Organigramm}
\end{figure}

\subsection{Projektplanung}
Für das PREN2 standen insgesamt 14 Wochen zu Verfügung, Um die Dokumentation 
abzuschliessen, stehen nach Abschluss des Semesters noch 2 Wochen zur 
Verfügung. Ausserdem kann das Gerät noch bis zum Wettbewerb optimiert werden. \\
Die Planung für das Projekt wurde mittels einer Excel-Tabelle gemacht. Die 
einzelnen Teilaufgaben wurden aus dem Konzept des PREN1 extrahiert und 
sinnvoll angeordnet und der jeweilige Aufwand abgeschätzt. So kann eine 
optimale Fertigung der Komponenten erzeugt werden um fortlaufen die 
Komponenten zusammenzufügen und zu Testen. In Anhang 
A
%\ref{app:project} 
findet sich die Excel-Tabelle welche verwendet wurde.
