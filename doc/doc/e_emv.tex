\subsection{EMV-Konzept}
\label{sec:e_emv}
Damit bei der Inbetriebnahme keine Probleme aufgrund von elektromagnetischen 
Einstrahlungen oder Störungen wegen Übersprechen auftreten, muss ein 
EMV-Konzept erarbeitet werden. Dieses wird in diesem Kapitel aufgezeigt. 

\noindent Für die Verbindung zwischen den einzelnen Platinen werden möglichst 
kurze Kabel verwendet. Es kommen ungeschirmte Flachbandkabel zum Einsatz. 
(Siehe \ref{sec:e_emv_opt} \nameref{sec:e_emv_opt}) Auf den einzelnen Platinen 
kommen Masseflächen zum Einsatz. Auf den Platinen für die Ansteuerung des 
DC-Motors und des BLDC-Motors werden Filter auf den Datenleitungen eingesetzt 
um hochfrequente Störungen zu dämpfen. Auf der BLDC-Platine besteht zudem die 
Möglichkeit ein spezielles Filter vom Typ BNX024 von Murata zu bestücken. 
Dieses beinhaltet einen Durchführungskondensator, eine Gleichtaktdrossel und 
einen Filterkondensator. Mit diesem können hochfrequente Anteile in der 
Speisung gefiltert werden. Weiter werden die Verbindungskabel zwischen 
Speisung und Gerät jeweils paarweise verdrillt. 

\subsubsection{Vorgehen beim Auftreten von Störungen}
\label{sec:e_emv_opt}
Treten während der Inbetriebnahme Störungen auf, muss das Gerät weiter 
entstört werden. Dazu können in einem ersten Schritt die Verbindungskabel in 
den vorgesehenen Kabelkanälen verlegt werden. Aufgrund der 
Aluminiumkonstruktion kann so eine gewisse Schirmwirkung erzielt werden. 
Weiter können anstelle der ungeschirmten Flachbandkabel geschirmte Kabel 
verwendet werden.  Ausserdem können in den Versorgungsleitungen 
Gleichtaktdrosseln in Form von Ferritringen installiert werden. Dadurch wird 
eine allfällige Abstrahlung und Störung über die Versorgungsleitungen 
reduziert. 

\noindent Da bei der Inbetriebnahme keine Störungen auftraten, wird auf obige 
optionale Erweiterungen verzichtet. 
