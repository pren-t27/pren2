\subsection{Softwaresubsysteme}
Nachfolgend sind die einzelnen Softwaresubsysteme und ihre Komponenten und Funktionalitäten beschrieben.
\subsubsection{Bluetoothmodul}
Das Bluetoothmodul setzt sich im Wesentlichen aus drei Komponenten zusammen. Diese Komponenten sind:
\begin{itemize}
	\item{\textbf{BluetoothSender:} Diese Klasse übernimmt die Befehlsübermittlung an das Freedomboard.}
	\item{\textbf{BluetoothReceiver:} Diese Klasse ist für das Empfangen der Statusnachrichten vom Freedomboard zuständig. Sie enthält eine Liste von BluetoothReceiverListener, deren Implementationen als Observer fungieren.}
	\item{\textbf{BluetoothController:} Der BluetoothController übernimmt die Initialisierung und Steuerung der einzelnen Komponenten und regelt deren Zusammenspiel. Er dient ausserdem als "'Durchlauferhitzer"' für BluetoothReceiverListener-Implementationen, damit diese sich beim BluetoothReceiver als Observer anmelden können.}
\end{itemize}

\subsubsection{Bilderkennung}
Um die Bilderkennung zu optimieren, wird statt des ganzen Bildes lediglich ein 
Teilausschnitt verwendet.  Der Bildausschnit wird mittels Maus-Drag 
ausgewählt. Der markierte Bereich wird danach auf dem Bildschirm dargestellt. \\

\noindent
Durch den kleineren Bildausschnitt, wird die Auswertung des Bildes schneller 
durchgeführt. Im folgenden sieht man wie der markierte Bereich auf dem 
Interface aussieht.\\

\begin{figure}[h!]          
	\centering             
	\includegraphics[width=1\textwidth]{fig/BildMitKorb_markiert.png}
	\caption{Bilderkennung - Ausschnitt}
	\label{fig:Bilderkennung_ausschnitt}        
\end{figure}

Der Ablauf der Bilderkennung funktioniert folgendermassen:
\begin{itemize}
	\item Der Bildausschnitt wird in ein schwarz/weiss Bild umgewandelt. \\
        Der Korb wird weiss, der Hintergrund schwarz dargestellt. 
	\item Ein Algorithmus scannt den Bildausschnitt Zeile für Zeile.
    \begin{itemize}
	    \item Es wird oben Links im Bildausschnitt gestartet.
	    \item Es wird Pixel für Pixel durchgelaufen und das erste weisse Pixel gespeichert.
	    \item Das Endpixel wird bestimmt, wenn auf das weisse Pixel ein schwazes Pixel folgt.
	    \item Von den Start und Endpunkte über die verschiedenen Zeilen wird der Mittelwert berechnet.
    \end{itemize}
\end{itemize}

\begin{figure}[h!]          
	\centering             
	\includegraphics[width=1\textwidth]{fig/Korberkennung_Schritte.png}
	\caption{Bilderkennung - Ablauf}
	\label{fig:Bilderkennung_ablauf}        
\end{figure}

\noindent
Mit dem gefundenen Mittelwert kann anhand von Geometrie der Winkel von der 
Maschine zum Korb bestimmt werden.  Es tritt ein Problem bei der 
Winkelbestimmung auf, falls der Korb sich links oder rechts am Rand befindet.  
Da der Blick der Kamera kegelförmig ist, wird nicht der gesammte Korb erfasst. 
Die äussere Hälfte des Korbs fehlt und dies führt dazu, dass der Winkel zu 
klein ist. Das Problem wird behoben, indem bei der Überschreitung eines 
Maximalwertes, der Winkel um ca. ein Grad erhöht wird.
