\subsection{Schnittstellenbeschreibung}
\subsubsection{Spezifikation BluetoothController-Klasse}

\lstsettingjava

\begin{table}[h!]
\begin{tabular}{|l|l|l|l|l|}
\hline 
Version & Datum & Autor & Bemerkung & Status \\ 
\hline 
1.0.0 & 26.03.2015 & SN & Initiale Fassung & freigegeben \\ 
\hline 
1.0.1 & 15.04.2015 & SN & Initiale Fassung & freigegeben \\ 
\hline 
\end{tabular} 
\caption{Steckbrief der Klasse BluetoothController}
\end{table}

\paragraph{Operationen und Datenstukturen}
 
Die Schnittstelle stellt folgende Methoden und Datentypen zur Verfügung:  \\
\begin{figure}[h!]          
	\centering             
	\includegraphics[width=1\textwidth]{../fig/Klassendiagramm Bluetoothmodul.png}
	\caption{Klassendiagramm Bluetoothmodul}
	\label{fig:Klassendiagramm Bluetoothmodul}        
\end{figure} \\
Details zu den Methoden und den verwendeten Datentypen sind in der JavaDoc festgehalten. \\
\paragraph{Einsatz, Abläufe, Voraussetzungen und Zusicherungen}
\begin{itemize}
	\item{Bevor Daten über die Schnittstelle ausgetauscht werden können, muss mittels initConnections(String serialPortName) ein gültiger serieller Port definiert werden. Der konfigurierte Port gilt für alle darauf folgenden command-Aufrufe. }
	\item{Ein Wechsel des seriellen Ports, d.h. Umkonfigurierung mit initConnections(String serialPortName), ist nun jederzeit möglich.}
	\item{Eine Klasse, die das Interface BluetoothReceiverListener implementiert, kann als Observer für empfangene Nachrichten des BluetoothReceivers fungieren.}
\end{itemize}
\paragraph{Aufbau und Konfiguration} 
Keine zusätzlichen Informationen. \\
\paragraph{Fehlerbehandlung}
Die Fehlerbehandlung wird über unchecked Exceptions realisiert. Details siehe JavaDoc. \\
\paragraph{Beispielverwendung}
Der folgende Codeausschnitt zeigt die Verwendung der Schnittstelle anhand einer beispielhaften Implementation BluetoothController: \\
\lstinputlisting[caption={Beispiel zur Verwendung des BluetoothControllers}]{../../sw/PrenManager/src/TestGUI/newClass.java}

\subsubsection{Spezifikation Bilderkennung}

\lstsettingjava

\begin{table}[h!]
	\begin{tabular}{|l|l|l|l|l|}
		\hline 
		Version & Datum & Autor & Bemerkung & Status \\ 
		\hline 
		1.0.0 & 26.03.2015 & SN & Initiale Fassung & freigegeben \\ 
		\hline 
		1.0.1 & 15.04.2015 & SN & Initiale Fassung & freigegeben \\ 
		\hline 
		2.0.1 & 07.05.2015 & PK & Anpassung für Erkennung-Klasse & freigegeben \\ 
		\hline 
	\end{tabular} 
	\caption{Steckbrief der Klasse Erkennung}
\end{table}

\paragraph{Operationen und Datenstukturen}

Die Schnittstelle stellt folgende Methoden und Datentypen zur Verfügung:  \\
\begin{figure}[h!]          
	\centering             
	\includegraphics[width=0.5\textwidth]{fig/Klassendiagramm_Erkennung.png}
	\caption{Klassendiagramm Bilderkennung}
	\label{fig:Klassendiagramm Bilderkennung}        
\end{figure} \\
Details zu den Methoden und den verwendeten Datentypen sind in der JavaDoc festgehalten. \\

\paragraph{Einsatz, Abläufe, Voraussetzungen und Zusicherungen}
\begin{itemize}
	\item{Die Erkennung Klasse wird wie folgt initialisiert: Erkennung erkenner = new Erkennung(); }
	\item{Es gibt die FUnktion processFrame(Mat frame) mit welchem ein Bild abgearbeitet wird und den Winkel als double zurückliefert.}
\end{itemize}

\paragraph{Aufbau und Konfiguration} 
Keine zusätzlichen Informationen. \\

\paragraph{Fehlerbehandlung}
Die Fehlerbehandlung wird über unchecked Exceptions realisiert. Details siehe JavaDoc. \\

\paragraph{Beispielverwendung}
Der folgende Codeausschnitt zeigt die Verwendung der Schnittstelle anhand einer beispielhaften Implementation BluetoothController: \\
\lstinputlisting[caption={Beispiel zur Verwendung der Bilderkennung}]{../../sw/PrenManager/src/TestGUI/Erkennung-Example.java}

\subsubsection{Spezifikation ImageGetter}

\lstsettingjava

\begin{table}[h!]
	\begin{tabular}{|l|l|l|l|l|}
		\hline 
		Version & Datum & Autor & Bemerkung & Status \\ 
		\hline 
		1.0.0 & 30.03.2015 & KW & Initiale Fassung & freigegeben \\ 
		\hline 
		1.0.1 & 16.04.2015 & KW & Initiale Fassung & freigegeben \\ 
		\hline 
		2.0.1 & 14.05.2015 & KW & Anpassung Bilderkennung & freigegeben \\ 
		\hline 
	\end{tabular} 
	\caption{Steckbrief der Klasse ImageGetter}
\end{table}

\paragraph{Operationen und Datenstukturen}

Die Schnittstelle stellt folgende Methoden und Datentypen zur Verfügung:  \\
\begin{figure}[h!]          
	\centering             
	\includegraphics[width=0.5\textwidth]{../fig/Klassendiagramm_ImageGetter.png}
	\caption{Klassendiagramm ImageGetter}
	\label{fig:Klassendiagramm ImageGetter}        
\end{figure} \\
Details zu den Methoden und den verwendeten Datentypen sind in der JavaDoc festgehalten. \\

\paragraph{Einsatz, Abläufe, Voraussetzungen und Zusicherungen}
\begin{itemize}
	\item{Die ImageGetter-Klasse wird wie folgt initialisiert: ImageHandler imgHandler = new ImageHandler(); }
	\item{Mittels startReceiving() der StreamGetter-Thread gestartet werden. Der StreamGetter steuert die Kamera an und speichert immer das aktuellste Bild. Dieses kann vom ImageHandler mittels getImage() geholt werden. Der ImageHandler stellt dieses Bild dann mit der eigenen getImage()-Funktion als Mat-Objekt zu Verfügung. Der Thread kann mittels terminateImages() beendet werden.}
\end{itemize}

\paragraph{Aufbau und Konfiguration} 
Keine zusätzlichen Informationen. \\

\paragraph{Fehlerbehandlung}
Die Fehlerbehandlung wird über unchecked Exceptions realisiert. Details siehe JavaDoc. \\

\paragraph{Beispielverwendung}
Der folgende Codeausschnitt zeigt die Verwendung der Schnittstelle anhand einer beispielhaften Implementation ImageGetter: \\
\lstinputlisting[caption={Beispiel zur Verwendung des ImageGetters}]{../../sw/PrenManager/src/TestGUI/TestImageGetter.java}
