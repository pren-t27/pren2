\section{Finanzen und Maschinenstunden}
Für das ganze Projekt steht ein Budget von CHF 600.- zur Verfügung. Da alle 
Blechteile von Daniel Mathis gefertigt werden, und für das Biegen nur eine 
symbolische Entschädigung von CHF 40.- anfällt, muss das Budget nicht 
vollständig ausgeschöpft werden. Um die Abrechnung zu erleichtern, wird 
Material, welches von Gruppenmitgliedern für PREN-ET (siehe Abschnitt 
\ref{sec:pren-et}) bestellt wird, direkt im jeweiligen Team abgerechnet. Da 
dieses Material unter den Teams verteilt wird, entspricht dies nicht den 
realen Kosten für das Team 27. 
\begin{table}[h!]
    \centering
    \begin{zebratabular}{llrl}
        \rowcolor{gray}
        Lieferant & 
            Bezeichnung & 
            Preis [CHF] & 
            Bezahlt \\
        Conrad Electronic & 
            Axial Rillenkugellager 51104 & 
            8.95 & 
            Yannik \\
        Microspot & 
            LOGITECH HD Webcam C525 & 
            41.85 & 
            Kevin\\
        Diverse & 
            Elektrotechnik & 
            287.00 &    
            Daniel W.\\
        Profiblech Kägiswil & 
            Blechteile biegen (Freundschaftspreis) & 
            34.40 & 
            Daniel M.\\
        & Rohmaterial Alublech 1m$^2$ & 
            25.00 & 
            gesponsert \\
        Arthur Weber AG & 
            Nieten / Aluprofile & 
            31.60   & 
            Daniel W. \\
        Neotexx &
            Neodym Magnete &
            32.30 &
            Daniel W. \\
        HSLU & 
            Schrittmotor QMot.eu 4218-35-10-027 & 
            30.00 & 
            - \\
        HSLU & 
            Servomotor Conrad TS-301 MGBB & 
            20.00 & 
            - \\
        Interdiscount & 
            Büromaterial & 
            15.90 & 
            Simon \\
        McPaperLand & 
            Büromaterial & 
            4.80 & 
            Simon \\
        & 
            TOTAL & 
            531.80 &
            \\
    \end{zebratabular}
    \caption{Finanzen PREN2}
\end{table}

\noindent
Von den 10h Fertigungszeit, die für das Projekt zur Verfügung stehen, 
werden 9.05 Stunden gebraucht. Folgende Teile werden gefertigt:

\begin{table}[h!]
    \centering
    \begin{zebratabular}{llll}
        \rowcolor{gray}
        Teilename              & Stückzahl & Material  & Aufwand in h \\
        Bodenplatte            & 1         & Aluminium & 3            \\
        Bein Adapter           & 3         & Aluminium & 2            \\
        Verbindung Bein-Platte & 3         & Aluminium & 1.8          \\
        Achshalter             & 1         & Aluminium & 0.75         \\
        Hülse Servo            & 2         & Aluminium & 0.25         \\
        Motorhalter            & 2         & Aluminium & 1.25         \\
        TOTAL                  &           &           & 9.05         \\
    \end{zebratabular}
    \caption{Fertigungszeit}
\end{table}

\noindent
Die drei Füsse werden mit einem 3D Drucke gedruckt. Dafür werden von den zu 
Verfügung stehenden 25 Stunden nur 0.8 Stunden benötigt. 
