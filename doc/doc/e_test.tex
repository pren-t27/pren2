\subsection{Test Elektronik}

\subsubsection{Test Energieversorgung}
\begin{table}[h!]
    \centering
    \begin{zebratabular}{p{0.11\textwidth}p{0.4\textwidth}p{0.3\textwidth}p{0.1\textwidth}}
        \rowcolor{gray} ID & Test & Erwartet & Ergebnis \\
        Supply-001   &
            Leerlaufspannung 48\si{\volt} &
            $48\si{\volt} \pm 1\si{\volt}$ &
            \boxed{} \\
        Supply-002   &
            Leerlaufspannung 5\si{\volt} &
            $5\si{\volt} \pm 0.1\si{\volt}$ &
            \boxed{} \\
    \end{zebratabular}
    \caption{Test Energieversorgung}
\end{table}
\FloatBarrier

\subsubsection{Test BLDC Treiber}
\begin{table}[h!]
    \centering
    \begin{zebratabular}{p{0.11\textwidth}p{0.4\textwidth}p{0.3\textwidth}p{0.1\textwidth}}
        \rowcolor{gray} ID & Test & Erwartet & Ergebnis \\
        BLDC-001 &
            Anlaufen &
            Anlaufen und Umschalten in Autokommutierung &
            \boxed{} \\
        BLDC-002 &
            Änderung Leistung &
            Die Leistung (PWM) kann via SPI eingestellt werden. &
            \boxed{} \\
        BLDC-003 &
            Drehzahlregelung &
            Die Drehzahl kann via SPI eingestellt werden und wird geregelt. &
            \boxed{} \\
    \end{zebratabular}
    \caption{Test BLDC Treiber}
\end{table}
\FloatBarrier

\subsubsection{Test Schrittmotortreiber}
\begin{table}[h!]
    \centering
    \begin{zebratabular}{p{0.11\textwidth}p{0.4\textwidth}p{0.3\textwidth}p{0.1\textwidth}}
        \rowcolor{gray} ID & Test & Erwartet & Ergebnis \\
        Step-001 &
            Nullen &
            Selbständiges Nullen &
            \boxed{} \\
        Step-002 &
            Position anfahren &
            Vorgegebene Position wird angefahren &
            \boxed{} \\
    \end{zebratabular}
    \caption{Test Schrittmotortreiber}
\end{table}
\FloatBarrier

\subsubsection{Test Gleichstrommotortreiber}
\begin{table}[h!]
    \centering
    \begin{zebratabular}{p{0.11\textwidth}p{0.4\textwidth}p{0.3\textwidth}p{0.1\textwidth}}
        \rowcolor{gray} ID & Test & Erwartet & Ergebnis \\
        DC-001 &
            Drehrichtung &
            Die Drehrichtung des DC Motors kann mittels DIR Signal vorgegeben werden. &
            \boxed{} \\
        DC-002 &
            Geschwindigkeit &
            Die Geschwindigkeit des DC Motors kann mittels PWM Signal beeinflusst werden. &
            \boxed{} \\
    \end{zebratabular}
    \caption{Test Gleichstrommotortreiber}
\end{table}
\FloatBarrier

%\subsubsection{Test Sensoren}
%\begin{table}[h!]
%    \centering
%    \begin{zebratabular}{p{0.11\textwidth}p{0.4\textwidth}p{0.3\textwidth}p{0.1\textwidth}}
%        \rowcolor{gray} ID & Test & Erwartet & Ergebnis \\
%        Sensor-001 &
%            Magnet bei Endanschlag Turm &
%            Signal low &
%            \boxed{} \\
%        Sensor-002 &
%            Magnet nicht bei Endanschlag Turm &
%            Signal high &
%            \boxed{} \\
%        Sensor-003 &
%            Beförderungsband eingefahren (Balllager voll) &
%            Signal high &
%            \boxed{} \\
%        Sensor-004 &
%            Beförderungsband nicht eingefahren (Balllager nicht voll) &
%            Signal low &
%            \boxed{} \\
%        Sensor-005 &
%            Beförderungsband ausgefahren (Balllager leer) &
%            Signal high &
%            \boxed{} \\
%        Sensor-006 &
%            Beförderungsband nicht ausgefahren (Balllager nicht leer) &
%            Signal low &
%            \boxed{} \\
%    \end{zebratabular}
%    \caption{Test Sensoren}
%\end{table}
%\FloatBarrier

\subsubsection{Test Kontrollplatine}
\begin{table}[h!]
    \centering
    \begin{zebratabular}{p{0.11\textwidth}p{0.4\textwidth}p{0.3\textwidth}p{0.1\textwidth}}
        \rowcolor{gray} ID & Test & Erwartet & Ergebnis \\
        FRDM-001 &
            Echo von Shell &
            gesendete Bytes werden korrekt zurückgesendet &
            \boxed{} \\
        FRDM-002 &
            Ansteuerung DC Motor &
            Drehzahl und Drehrichtung des DC Motors kann via Shell variiert werden &
            \boxed{} \\
        FRDM-003 &
            Ansteuerung BLDC Motor &
            BLDC Motor kann via Shell angesteuert werden. &
            \boxed{} \\
        FRDM-004 &
            Nullung Schrittmotor &
            Schrittmotor kann via Shell genullt werden. &
            \boxed{} \\
        FRDM-005 &
            Ansteuerung Schrittmotor &
            Schrittmotor kann via Shell an eine gegebene Position gefahren werden. 
            \boxed{} \\
    \end{zebratabular}
    \caption{Test Kontrollplatine}
\end{table}
\FloatBarrier

\subsubsection{Test Bluetooth Modul}
\begin{table}[h!]
    \centering
    \begin{zebratabular}{p{0.11\textwidth}p{0.4\textwidth}p{0.3\textwidth}p{0.1\textwidth}}
        \rowcolor{gray} ID & Test & Erwartet & Ergebnis \\
        BT-001 &
            Korrekte Übertragung eines Bytes in Richtung Computer $\to$ FRDM-KL25Z &
            Byte korrekt übertragen &
            \boxed{} \\
        BT-002 &
            Korrekte Übertragung eines Bytes in Richtung FRDM-KL25Z $\to$ Computer &
            Byte korrekt übertragen &
            \boxed{} \\
    \end{zebratabular}
    \caption{Test Bluetooth Modul}
\end{table}
\FloatBarrier

