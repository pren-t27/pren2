\addcontentsline{toc}{section}{\protect\numberline{}Management Summary}
\section*{Management Summary}
Die Aufgabe im Modul PREN2 besteht darin, das Konzept 
(autonomer Ballwerfer) des PREN1 umzusetzen. Das Konzept
des PREN1 wird konkret umgesetzt, indem die jeweiligen
Abteilungen die Teilkomponenten erstellen und zusammenfügen.
Am Ende wird das Gerät in einem Wettbewerb gegen andere
Geräte antreten. Um das Konzept umzusetzen arbeiten die 
verschiedenen Disziplinen (Maschinenbau, Elektrotechnik, Informatik) 
unabhängig. Lediglich zu Beginn werden bei einigen Komponenten die 
benötigten Schnittstellen klar definiert und Vorgaben erstellt,
welche eingehalten werden müssen. Nach der Fertigung der
Komponenten werden diese fortlaufend zusammengebaut und Tests 
unterzogen. So kann die Funktionalität gewährleistet werden. 
Die Informatik und Elektrotechnik arbeiten im Bereich der 
Programmierung eng zusammen und testen die Software während 
der Erstellung. So ist das Zusammenbauen des Endprodukts schnell 
und allfällige kleine Mängel werden in den Integrationstests 
sichtbar. Diese Mängel können schnell behoben werden. Nach der 
Umsetzung des Konzepts haben wir einen funktionierenden 
autonomen Ballwerfer der die geforderte Aufgabe im definierten 
Umfang erledigt.
