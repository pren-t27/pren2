\section{Übersicht Maschinentechnik}
Die Konstruktion besteht hauptsächlich aus Aluminium-Blechteilen. Einige 
Komponenten mussten aus konstruktiven Gründen als Frästeile ausgeführt werden.
Um ein geringeres Gewicht zu 
erreichen, werden die Blechteile mit Aussparungen an nicht erforderlichen 
Flächen versehen. Da dies eine Schwächung der Stabilität mit sich bringt werden die Aussparungen, wie im Flugzeugbau üblich, mit gebogenen Innenkanten ausgeführt. 

Foto



\subsection{Herstellung mechanische Komponenten}
Die konzeptionellen CAD-Zeichnungen werden ab Semesterwoche 2 weiter 
ausgearbeitet, abgewickelt und die Fertigungszeichnungen erstellt. Hierbei 
müssen hauptsächlich Details wie Bohrungen für die Nieten angebracht und 
einige konstruktive Anpassungen vorgenommen werden. 
Ab Semesterwoche 3 können die ersten Teile an der Fräsmaschine im Elektrotechniklabor produziert werden, 
wobei parallel  hierzu die restlichen Komponenten am CAD fertiggestellt werden.
Um die gebogenen Innenkanten der Aussparungen zu realisieren, werden die gefrästen Blechteile mit der Handpresse und eigens hierfür hergestellten Werkzeugen gefertigt.

FOTO

In Semesterwoche 4 werden die Grundplatte zur spanenden Fertigung an der HSLU 
in Auftrag gegeben.
In Semesterwoche 5 werden zusätzlich einige Teile zum spanenden Herstellen, 
sowie zum 3D-Drucken in Auftrag gegeben. Ebenfalls wird eine 
Rohmaterialbestellung abgegeben.
Die in Auftrag gegebenen Teile können in Semesterwoche 6 abgeholt werden, 
wobei das  Rohmaterial fälschlicherweise aus Stahl, anstatt aus Aluminium, 
bestellt wurde. Eine neue Bestellung des richtigen Rohmaterials wurde abgesetzt.
In Semesterwoche 7 können die bestellten Teile sowie zum biegen extern in Auftrag gegebene Teile abgeholt werden. Des weiteren wird mit dem Zusammenbau der einzelnen Komponenten begonnen. Aufgrund eines beim Biegen entstandenen Verzugs einzelner Bauteile und einiger Konstruktiver Ungenauigkeiten müssen diverse Bohrungen durch Feilen von Hand nachgebessert werden. Durch Blindnieten kann die Konstruktion bis zum endgültigen Vernieten aufgebaut werden.