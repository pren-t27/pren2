\section{Übersicht Maschinentechnik}
Die Konstruktion besteht hauptsächlich aus Aluminium-Blechteilen. Einige 
Komponenten mussten aus konstruktiven Gründen als Frästeile ausgeführt werden.
Um eine möglichst gute Stabilität bei verhältnismässig geringem Gewicht zu 
erreichen, wurden die Blechteile mit Aussparungen an nicht erforderlichen 
Flächen versehen.

\subsection{Herstellung mechanische Komponenten}
Die konzeptionellen CAD-Zeichnungen wurden ab Semesterwoche 2 weiter 
ausgearbeitet, abgewickelt und die Fertigungszeichnungen erstellt. Hierbei 
mussten hauptsächlich Details wie Bohrungen für die Nieten angebracht und 
einige konstruktive Anpassungen vorgenommen werden. Ab Semesterwoche 3 konnten 
die ersten Teile an der Fräsmaschine im Elektrotechniklabor produziert werden, 
wobei parallel  hierzu die restlichen Komponenten am CAD fertiggestellt wurden.
In Semesterwoche 4 wurde die Grundplatte zur spanenden Fertigung an der HSLU 
in Auftrag gegeben.
In Semesterwoche 5 wurden zusätzlich einige Teile zum spanenden Herstellen, 
sowie zum 3D-Drucken in Auftrag gegeben. Ebenfalls wurde eine 
Rohmaterialbestellung abgegeben.
Die in Auftrag gegebenen Teile konnten in Semesterwoche 6 abgeholt werden, 
wobei das  Rohmaterial fälschlicherweise aus Stahl, anstatt aus Aluminium, 
bestellt wurde. Eine neue Bestellung des richtigen Rohmaterials wurde abgesetzt.
