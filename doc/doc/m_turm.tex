\subsection{Turm}
Der Turm dient als Verbindungselement des Balllager und des grösseren 
Zahnrades der Drehvorrichtung. Ausgeführt wird die gesamte Konstruktion aus 
Aluminiumblech. Der Platz im Inneren wird zum Verstauen der 
Elektronikkomponenten verwendet. Hierfür wird die vordere Abdeckplatte statt mit Nieten, mit Schrauben befestigt. Die Muttern werden auf der Innenseite des Turms mit 2 Komponenten Klebstoff befestigt.