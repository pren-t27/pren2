\subsection{Fachgruppe Elektrotechnik}
\label{sec:pren-et}
Elektrotechnik-Studierenden aus mehreren Gruppen haben sich
zusammengeschlossen um gemeinsame Probleme anzugehen. Dabei handelt es sich
um die benötigte Hard- und Software, um Motoren anzusteuern
und gegebenenfalls zu regeln. In diesem Zusammenschluss werden drei Gruppen
gebildet, um Lösungen für DC-, Stepper- und Brushless-Motoren auszuarbeiten.
Die Idee besteht darin, dass nicht jede Gruppe für dasselbe Problem wo
möglich denselben Lösungsansatz verfolgt, sondern die Ressourcen kombiniert,
Synergien nutzt, um eine bessere Lösung zu erarbeiten. Auf diese Weise kann
das teamübergreifende Arbeiten im Rahmen des PREN erlernt und
geübt werden. Somit wird Idee der Interdisziplinarität im erweiterten Sinn
Rechnung getragen. Die Gruppen und deren Mitglieder sind in 
\autoref{tab:pren-et-members} aufgeführt.
\begin{table}[h!]
    \centering
    \begin{zebratabular}{lllccc}
        \rowcolor{gray} 
        Team & Mitglied         & Github    & DC          & BLDC        & Stepper     \\
        27   & Daniel Winz      & daniw     &             & \textbullet & \textbullet \\
        32   & Yves Studer      & ystuder   &             & \textbullet &             \\
        33   & Flavio Kreiliger & Flavinsky & \textbullet &             & \textbullet \\
        38   & Bettina Wyss     & BettyET   &             &             & \textbullet \\
        39   & Ervin Mazlagi\'c & ninux     & \textbullet &             &             \\
    \end{zebratabular}
    \caption{Übersicht der PREN-ET Projektgruppen}
    \label{tab:pren-et-members}
\end{table}
Für den Austausch und die Ablage von Daten und Unterlagen wird die Plattform 
Github ausgewählt, da damit via Git\footnote{Verteiltes 
Versionskontrollsystem} versioniert werden kann. Dazu wird auf Github die 
Organisation PREN-ET gegründet. Diese ist unter 
\url{https://github.com/PREN-ET} einsehbar. für die einzelnen Projekte werden 
Repositories angelegt. Diese sind in \autoref{tab:pren-et-repos} aufgeführt. 
\begin{table}[h!]
    \centering
    \begin{zebratabular}{p{0.09\textwidth}p{0.37\textwidth}p{0.4\textwidth}}
        \rowcolor{gray} 
        Repository  & Link         & Beschreibung \\
        info        & \url{https://github.com/PREN-ET/info}    & Allgemeine Informationen zur Organisation von PREN-ET \\
        doc         & \url{https://github.com/PREN-ET/doc}     & Dokumentation \\
        dc          & \url{https://github.com/PREN-ET/dc}      & Treiber für Gleichstrommotoren \\
        bldc        & \url{https://github.com/PREN-ET/bldc}    & Treiber für Brushless Motoren \\
        stepper     & \url{https://github.com/PREN-ET/stepper} & Treiber für Schrittmotoren \\
        frdm        & \url{https://github.com/PREN-ET/frdm}    & Beispiele zur Ansteuerung mittels FRDM-KL25Z \\
    \end{zebratabular}
    \caption{Übersicht der PREN-ET Repositories}
    \label{tab:pren-et-repos}
\end{table}
